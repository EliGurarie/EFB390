\PassOptionsToPackage{unicode=true}{hyperref} % options for packages loaded elsewhere
\PassOptionsToPackage{hyphens}{url}
%
\documentclass[9pt,ignorenonframetext,aspectratio=169]{beamer}
\usepackage{pgfpages}
\setbeamertemplate{caption}[numbered]
\setbeamertemplate{caption label separator}{: }
\setbeamercolor{caption name}{fg=normal text.fg}
\beamertemplatenavigationsymbolsempty
% Prevent slide breaks in the middle of a paragraph:
\widowpenalties 1 10000
\raggedbottom
\setbeamertemplate{part page}{
\centering
\begin{beamercolorbox}[sep=16pt,center]{part title}
  \usebeamerfont{part title}\insertpart\par
\end{beamercolorbox}
}
\setbeamertemplate{section page}{
\centering
\begin{beamercolorbox}[sep=12pt,center]{part title}
  \usebeamerfont{section title}\insertsection\par
\end{beamercolorbox}
}
\setbeamertemplate{subsection page}{
\centering
\begin{beamercolorbox}[sep=8pt,center]{part title}
  \usebeamerfont{subsection title}\insertsubsection\par
\end{beamercolorbox}
}
\AtBeginPart{
  \frame{\partpage}
}
\AtBeginSection{
  \ifbibliography
  \else
    \frame{\sectionpage}
  \fi
}
\AtBeginSubsection{
  \frame{\subsectionpage}
}
\usepackage{lmodern}
\usepackage{amssymb,amsmath}
\usepackage{ifxetex,ifluatex}
\usepackage{fixltx2e} % provides \textsubscript
\ifnum 0\ifxetex 1\fi\ifluatex 1\fi=0 % if pdftex
  \usepackage[T1]{fontenc}
  \usepackage[utf8]{inputenc}
  \usepackage{textcomp} % provides euro and other symbols
\else % if luatex or xelatex
  \usepackage{unicode-math}
  \defaultfontfeatures{Ligatures=TeX,Scale=MatchLowercase}
\fi
% use upquote if available, for straight quotes in verbatim environments
\IfFileExists{upquote.sty}{\usepackage{upquote}}{}
% use microtype if available
\IfFileExists{microtype.sty}{%
\usepackage[]{microtype}
\UseMicrotypeSet[protrusion]{basicmath} % disable protrusion for tt fonts
}{}
\IfFileExists{parskip.sty}{%
\usepackage{parskip}
}{% else
\setlength{\parindent}{0pt}
\setlength{\parskip}{6pt plus 2pt minus 1pt}
}
\usepackage{hyperref}
\hypersetup{
            pdftitle={Что такое осень? Что такое весна?},
            pdfauthor={Elie Gurarie - University of Maryland},
            pdfborder={0 0 0},
            breaklinks=true}
\urlstyle{same}  % don't use monospace font for urls
\newif\ifbibliography
\usepackage{color}
\usepackage{fancyvrb}
\newcommand{\VerbBar}{|}
\newcommand{\VERB}{\Verb[commandchars=\\\{\}]}
\DefineVerbatimEnvironment{Highlighting}{Verbatim}{commandchars=\\\{\}}
% Add ',fontsize=\small' for more characters per line
\usepackage{framed}
\definecolor{shadecolor}{RGB}{248,248,248}
\newenvironment{Shaded}{\begin{snugshade}}{\end{snugshade}}
\newcommand{\AlertTok}[1]{\textcolor[rgb]{0.94,0.16,0.16}{#1}}
\newcommand{\AnnotationTok}[1]{\textcolor[rgb]{0.56,0.35,0.01}{\textbf{\textit{#1}}}}
\newcommand{\AttributeTok}[1]{\textcolor[rgb]{0.77,0.63,0.00}{#1}}
\newcommand{\BaseNTok}[1]{\textcolor[rgb]{0.00,0.00,0.81}{#1}}
\newcommand{\BuiltInTok}[1]{#1}
\newcommand{\CharTok}[1]{\textcolor[rgb]{0.31,0.60,0.02}{#1}}
\newcommand{\CommentTok}[1]{\textcolor[rgb]{0.56,0.35,0.01}{\textit{#1}}}
\newcommand{\CommentVarTok}[1]{\textcolor[rgb]{0.56,0.35,0.01}{\textbf{\textit{#1}}}}
\newcommand{\ConstantTok}[1]{\textcolor[rgb]{0.00,0.00,0.00}{#1}}
\newcommand{\ControlFlowTok}[1]{\textcolor[rgb]{0.13,0.29,0.53}{\textbf{#1}}}
\newcommand{\DataTypeTok}[1]{\textcolor[rgb]{0.13,0.29,0.53}{#1}}
\newcommand{\DecValTok}[1]{\textcolor[rgb]{0.00,0.00,0.81}{#1}}
\newcommand{\DocumentationTok}[1]{\textcolor[rgb]{0.56,0.35,0.01}{\textbf{\textit{#1}}}}
\newcommand{\ErrorTok}[1]{\textcolor[rgb]{0.64,0.00,0.00}{\textbf{#1}}}
\newcommand{\ExtensionTok}[1]{#1}
\newcommand{\FloatTok}[1]{\textcolor[rgb]{0.00,0.00,0.81}{#1}}
\newcommand{\FunctionTok}[1]{\textcolor[rgb]{0.00,0.00,0.00}{#1}}
\newcommand{\ImportTok}[1]{#1}
\newcommand{\InformationTok}[1]{\textcolor[rgb]{0.56,0.35,0.01}{\textbf{\textit{#1}}}}
\newcommand{\KeywordTok}[1]{\textcolor[rgb]{0.13,0.29,0.53}{\textbf{#1}}}
\newcommand{\NormalTok}[1]{#1}
\newcommand{\OperatorTok}[1]{\textcolor[rgb]{0.81,0.36,0.00}{\textbf{#1}}}
\newcommand{\OtherTok}[1]{\textcolor[rgb]{0.56,0.35,0.01}{#1}}
\newcommand{\PreprocessorTok}[1]{\textcolor[rgb]{0.56,0.35,0.01}{\textit{#1}}}
\newcommand{\RegionMarkerTok}[1]{#1}
\newcommand{\SpecialCharTok}[1]{\textcolor[rgb]{0.00,0.00,0.00}{#1}}
\newcommand{\SpecialStringTok}[1]{\textcolor[rgb]{0.31,0.60,0.02}{#1}}
\newcommand{\StringTok}[1]{\textcolor[rgb]{0.31,0.60,0.02}{#1}}
\newcommand{\VariableTok}[1]{\textcolor[rgb]{0.00,0.00,0.00}{#1}}
\newcommand{\VerbatimStringTok}[1]{\textcolor[rgb]{0.31,0.60,0.02}{#1}}
\newcommand{\WarningTok}[1]{\textcolor[rgb]{0.56,0.35,0.01}{\textbf{\textit{#1}}}}
\setlength{\emergencystretch}{3em}  % prevent overfull lines
\providecommand{\tightlist}{%
  \setlength{\itemsep}{0pt}\setlength{\parskip}{0pt}}
\setcounter{secnumdepth}{0}

% set default figure placement to htbp
\makeatletter
\def\fps@figure{htbp}
\makeatother

\usepackage[T1, T2A]{fontenc}
\definecolor{links}{HTML}{2A1B81}
\hypersetup{colorlinks,linkcolor=,urlcolor=links}

\setbeamertemplate{items}[ball] 
\setbeamertemplate{blocks}[rounded][shadow=true] 
\setbeamertemplate{navigation symbols}{}
\setbeamertemplate{footline}{}

\setbeamersize{text margin left=1cm,text margin right=1cm} 

\usepackage{alltt}
\usepackage{multicol}
\usepackage{multirow}
\usepackage{tabulary}

% For absolute positioning of text (e.g. little footlines)
\usepackage[absolute,overlay]{textpos} 
	
% Color commands
  \definecolor{darkblue}{rgb}{0,0,.3}
  \definecolor{verylightblue}{rgb}{.9,.95,.99}
  \definecolor{lightblue}{rgb}{.5,.6,.95}
	\definecolor{green1}{rgb}{0,.5,.4}
	\definecolor{cream1}{rgb}{.97,.97,.83}
	\definecolor{deepcream}{rgb}{.2,.2,.05}
	\definecolor{emphasis}{rgb}{0,.5,.1}
	\definecolor{brightred}{rgb}{.95,.6,.6}
	\definecolor{white}{rgb}{1,1,1}	
	\definecolor{yellow}{rgb}{1,1,0}	
	\definecolor{darkgreen}{rgb}{0,0.2,0}
	\definecolor{darkorange}{rgb}{0.4,0.3,0}
	\definecolor{darkred}{rgb}{0.7,0,0}
	\definecolor{lightgray}{gray}{0.9}
	\definecolor{lightred}{rgb}{1,.5,.5}
	
	\newcommand{\drd}{\textcolor{darkred}}
	\newcommand{\rd}{\textcolor{red}}
	\newcommand{\lrd}{\textcolor{lightred}}		
	\newcommand{\bl}{\textcolor{blue}}
	\newcommand{\gr}{\textcolor{darkgreen}}
	\newcommand{\blk}{\textcolor{black}}
	\newcommand{\wt}{\textcolor{white}}
	\newcommand{\crm}{\textcolor{cream1}}
	
	\setbeamerfont{small}{size=\small}
	\setbeamerfont{smaller}{size=\footnotesize}
	\setbeamerfont{tiny}{size=\tiny}
	\setbeamerfont{teeny}{size=\scriptsize}
	
	
	\setbeamercolor{uppercol}{fg=cream1,bg=darkorange}
	\setbeamercolor{lowercol}{fg=black,bg=cream1}
  %\setbeamercolor{boxcolor}{fg=cream1,bg=green1}
	%\setbeamercolor{boxcolor1}{fg=white,bg=darkorange}
	%\setbeamercolor{boxcolor2}{fg=black,bg=cream1}
  \setbeamercolor{titlelike}{bg = darkgreen, fg = white}
  \setbeamercolor{block title}{bg = darkorange, fg = white}
	\setbeamercolor{block body}{bg = cream1, fg = black}
	  
	\newcommand{\BackH}[1]{\setbeamertemplate{background canvas}{\includegraphics[height=\paperheight]{#1}}}
	\newcommand{\BackW}[1]{\setbeamertemplate{background canvas}{\includegraphics[width=\paperwidth]{#1}}}
	\newcommand{\BackPlain}{\setbeamertemplate{background canvas}{}}
	
	\newcommand{\PlotH}[2][1]{\includegraphics[height=#1\textheight]{#2}}
	\newcommand{\PlotW}[2][1]{\includegraphics[width=#1\textwidth]{#2}}

\newcommand{\MultiPlotH}[3][1]{\includegraphics[height=#1\textheight, page=#2]{#3}}
\newcommand{\MultiPlotW}[3][1]{\includegraphics[width=#1\textwidth, page=#2]{#3}}

\setbeamercolor{itemize item}{fg = black}
\setbeamercolor{itemize subitem}{fg = black}
\setbeamercolor{enumerate item}{fg = black}
\setbeamercolor{enumerate subitem}{fg = black}


% itemization commands
%
	\newcommand{\bi}{\begin{itemize}}
	\newcommand{\ei}{\end{itemize}}
	\newcommand{\I}{\item}
	\newcommand{\ben}{\begin{enumerate}}
	\newcommand{\een}{\end{enumerate}}
	
	\newcommand{\bcol}{\begin{columns}}
	\newcommand{\col}[1][.5]{\column{#1\textwidth} }
	\newcommand{\ecol}{\end{columns}}

	\newcommand{\bc}{\begin{center}}
	\newcommand{\ec}{\end{center}}
	
	\newcommand{\beq}{\begin{equation*}}
	\newcommand{\eeq}{\end{equation*}}
	
	\newcommand{\beqrray}{\begin{eqnarray*}}
	\newcommand{\eeqrray}{\end{eqnarray*}}
	
	\newcommand{\bbb}[1]{\begin{beamerboxesrounded}[upper=uppercol,lower=lowercol,shadow=false]{#1}}			
	\newcommand{\ebb}{\end{beamerboxesrounded}}

	\newcommand{\nd}{\vspace{.1in}}
	\newcommand{\nup}{\vspace{-.1in}}
	\newcommand{\nl}{\hspace{-.1in}}
	\newcommand{\nr}{\hspace{+.1in}}
	
% Creates little textblocks!?
	\newcommand\FrameText[1]{
	  \begin{textblock*}{\paperwidth}(0pt,1pt)
	    \raggedright #1
	  \end{textblock*}}
	

\usepackage[scaled=0.85]{PTMono}

\title{Что такое осень? Что такое весна?}
\providecommand{\subtitle}[1]{}
\subtitle{Analysis of Phenological Data with the \texttt{ebfb} package}
\author{Elie Gurarie - University of Maryland}
\date{November 5, 2017}

\begin{document}
\frame{\titlepage}

\begin{frame}[fragile]{The \texttt{ebfb} package}
\protect\hypertarget{the-ebfb-package}{}

\bcols
\col{.5}

\bi

\I The \texttt{ebfb} package is an in-house package for the analysis of
the phenological data in \texttt{R}. \I In this presentation, I show a
few possible analyses WITH ALL THE CODE.\\
\I If you install the package, you can easily recreate the figrus and
replicate the results \ei

\col{.5}

\bi

\I Пакет \texttt{ebfb} - данные и функции для рассматривания, обработки
и анализа фенологический данных в (безплатном!) прогамме \texttt{R}.
\I В этом докладе, я демонстрирую пару возможных анализов и ВКЛЮЧАЮ
СКРИПТЫ. \I Если вы инсталируете пакет, то сможете запросто поврторить
анализ и создать соответственные картини. \ei 

\ecols

\end{frame}

\begin{frame}[fragile]{loading}
\protect\hypertarget{loading}{}

\bcols
\col{.5}

Install and load library \col{.5}

Инсталироавать и загружать библиотеку

\ecols

\begin{Shaded}
\begin{Highlighting}[]
\KeywordTok{library}\NormalTok{(ebfb)}
\end{Highlighting}
\end{Shaded}

\end{frame}

\begin{frame}{Data sets \textbar{} Данные}
\protect\hypertarget{data-sets-}{}

\bcols
\col{.5}

The package - first and foremost - contains the relevant data \col{.5} В
пакете, в первую очередь, держатся данные: \ecols

\small

\end{frame}

\begin{frame}[fragile]{Data: \texttt{sites}}
\protect\hypertarget{data-sites}{}

\bcols
\col{.5}

Table of site information: Name, Coordinates, N. Observations \col{.5}
Таблица мест наблюдений (в основном, ООПТ). Загружается одной строкой:
\ecols

\begin{Shaded}
\begin{Highlighting}[]
\KeywordTok{data}\NormalTok{(sites)}
\end{Highlighting}
\end{Shaded}

\bcols
\col{.5}

Detailed explanation: \col{.5} Подробное описание: \ecols

\begin{Shaded}
\begin{Highlighting}[]
\NormalTok{?sites}
\end{Highlighting}
\end{Shaded}

\end{frame}

\begin{frame}[fragile]{Data: \texttt{sites}}
\protect\hypertarget{data-sites-1}{}

\bcols
\col{.5}

The first few rows:

\col{.5}

Первые ряды: \ecols

\scriptsize

\begin{Shaded}
\begin{Highlighting}[]
\KeywordTok{head}\NormalTok{(sites)}
\end{Highlighting}
\end{Shaded}

\normalsize В сумме: X локаций

\end{frame}

\begin{frame}[fragile]{Data: \texttt{sites}}
\protect\hypertarget{data-sites-2}{}

\bcols

\col{.5} Map of sites: \col{.5} Карта: \ecols

\small

\begin{Shaded}
\begin{Highlighting}[]
\KeywordTok{require}\NormalTok{(maps); }\KeywordTok{require}\NormalTok{(mapdata)}
\KeywordTok{map}\NormalTok{(}\StringTok{"worldHires"}\NormalTok{, }\DataTypeTok{xlim =} \KeywordTok{c}\NormalTok{(}\DecValTok{2}\NormalTok{,}\DecValTok{160}\NormalTok{), }\DataTypeTok{ylim =} \KeywordTok{c}\NormalTok{(}\DecValTok{40}\NormalTok{,}\DecValTok{75}\NormalTok{))}
\KeywordTok{points}\NormalTok{(sites}\OperatorTok{$}\NormalTok{East, sites}\OperatorTok{$}\NormalTok{North, }\DataTypeTok{col=}\StringTok{"darkblue"}\NormalTok{, }\DataTypeTok{pch =} \DecValTok{19}\NormalTok{)}\ErrorTok{)}
\end{Highlighting}
\end{Shaded}

\bcols\col{.2}\col{.6}

\col{.2}\ecols

\end{frame}

\begin{frame}[fragile]{Data: \texttt{phenodata}}
\protect\hypertarget{data-phenodata}{}

\bcols
\col{.5}

This is the main dataset that contains all the of the phenological
events. To load the data, type: \col{.5} Главные данные всех
фено-явлений. Загружается одной строкой: \ecols

\begin{Shaded}
\begin{Highlighting}[]
\KeywordTok{data}\NormalTok{(phenodata)}
\end{Highlighting}
\end{Shaded}

\bcols
\col{.5}

Detailed explanation: \col{.5} Подробное описание: \ecols

\begin{Shaded}
\begin{Highlighting}[]
\NormalTok{?phenodata}
\end{Highlighting}
\end{Shaded}

\end{frame}

\begin{frame}[fragile]{\texttt{phenodata}}
\protect\hypertarget{phenodata}{}

\bcols\col{.5}The first few rows: \col{.5}Первые ряды:\ecols

\begin{Shaded}
\begin{Highlighting}[]
\KeywordTok{head}\NormalTok{(phenodata, }\DecValTok{5}\NormalTok{)}
\end{Highlighting}
\end{Shaded}

\scriptsize

\small
\bcols\col{.5}

Each row represents a single \textbf{event} for a single
\textbf{species} at a single \textbf{site} in a single \textbf{year}.
The main response variable is \textbf{day} (of year)

\col{.5}

Каждый ряд отражает одно \texttt{явление}, для одного \texttt{вида}, в
одном \texttt{месте}, в одном \texttt{году}. Главная переменная -
\texttt{День} (года).\\
\ecols

\end{frame}

\begin{frame}[fragile]{\texttt{phenodata\$Group}}
\protect\hypertarget{phenodatagroup}{}

\footnotesize

\begin{Shaded}
\begin{Highlighting}[]
\KeywordTok{table}\NormalTok{(phenodata}\OperatorTok{$}\NormalTok{Group)}
\end{Highlighting}
\end{Shaded}

\nup

\begin{Shaded}
\begin{Highlighting}[]
\KeywordTok{barplot}\NormalTok{(}\KeywordTok{sort}\NormalTok{(}\KeywordTok{table}\NormalTok{(phenodata}\OperatorTok{$}\NormalTok{Group)), }\DataTypeTok{horiz =} \OtherTok{TRUE}\NormalTok{, }\DataTypeTok{las =} \DecValTok{1}\NormalTok{)}
\end{Highlighting}
\end{Shaded}

\bcols\col{.7}

\col{.3}
\normalsize

Много растений!

Мало слизняков! \ecols

\end{frame}

\begin{frame}[fragile]{\texttt{phenodata\$Events}}
\protect\hypertarget{phenodataevents}{}

\bcols\col{.5} Most common events: \col{.5} Самые многочисленные явления
в данных:\ecols

\scriptsize

\begin{Shaded}
\begin{Highlighting}[]
\KeywordTok{sort}\NormalTok{(}\KeywordTok{table}\NormalTok{(}\KeywordTok{subset}\NormalTok{(phenodata, Group }\OperatorTok{==}\StringTok{ "Plants"}\NormalTok{)}\OperatorTok{$}\NormalTok{Event), }\DataTypeTok{decreasing =} \OtherTok{TRUE}\NormalTok{)[}\DecValTok{1}\OperatorTok{:}\DecValTok{10}\NormalTok{]}
\KeywordTok{sort}\NormalTok{(}\KeywordTok{table}\NormalTok{(}\KeywordTok{subset}\NormalTok{(phenodata, Group }\OperatorTok{==}\StringTok{ "Birds"}\NormalTok{)}\OperatorTok{$}\NormalTok{Event), }\DataTypeTok{decreasing =} \OtherTok{TRUE}\NormalTok{)[}\DecValTok{1}\OperatorTok{:}\DecValTok{10}\NormalTok{]}
\end{Highlighting}
\end{Shaded}

\end{frame}

\begin{frame}[fragile]{\texttt{phenodata\$Species}}
\protect\hypertarget{phenodataspecies}{}

\bcols

\col{.6}
\footnotesize

\begin{Shaded}
\begin{Highlighting}[]
\KeywordTok{table}\NormalTok{(}\KeywordTok{subset}\NormalTok{(phenodata, Group }\OperatorTok{==}\StringTok{ "Plants"}\NormalTok{)}\OperatorTok{$}\NormalTok{Species)}
\end{Highlighting}
\end{Shaded}

\normalsize 989 plants in total

\col{.4}

Most common plants:

Наиболее многочисленные растения:

\scriptsize

\ecols

\end{frame}

\begin{frame}{Question 1 \textbar{} Вопрос 1}
\protect\hypertarget{question-1--1}{}

\bcols
\col{.5}

Has flowering time of birches changed in the past century? \col{.5}
Изменился ли срок распускания цветов березы за последнее столетие?
\ecols

\bc \PlotW{.5}{floweringbirch} \ec

\end{frame}

\begin{frame}[fragile]{First flowering time of birch / Распускание
цветов березы \emph{Betula pendula}}
\protect\hypertarget{first-flowering-time-of-birch---}{}

\small

\begin{Shaded}
\begin{Highlighting}[]
\NormalTok{birches <-}\StringTok{ }\KeywordTok{subset}\NormalTok{(phenodata, Species }\OperatorTok{==}\StringTok{ "Betula pendula"} \OperatorTok{&}\StringTok{ }\NormalTok{Event }\OperatorTok{==}\StringTok{ "onset of blooming"}\NormalTok{)}
\KeywordTok{ddply}\NormalTok{(birches, }\StringTok{"Site"}\NormalTok{, summarize, }\DataTypeTok{n =} \KeywordTok{length}\NormalTok{(Day), }
            \DataTypeTok{firstyear =} \KeywordTok{min}\NormalTok{(Year), }\DataTypeTok{lastyear =} \KeywordTok{max}\NormalTok{(Year),}
            \DataTypeTok{mean.day =} \KeywordTok{mean}\NormalTok{(Day), }\DataTypeTok{sd.day =} \KeywordTok{sd}\NormalTok{(Day)) }\OperatorTok\StringTok{ }\KeywordTok{arrange}\NormalTok{(mean.day)}
\end{Highlighting}
\end{Shaded}

\normalsize

Earliest average (April 24) in Браславские Озера, Беларусь.

Latest (May 25) in Столбы, Сибирь.

\end{frame}

\begin{frame}[fragile]{Change in time \textbar{} Изменение по времени -
пример}
\protect\hypertarget{change-in-time------}{}

\bc
\footnotesize

\begin{Shaded}
\begin{Highlighting}[]
\KeywordTok{with}\NormalTok{(}\KeywordTok{subset}\NormalTok{(birches, Site }\OperatorTok{==}\StringTok{ "Centralno-Chernozemnyj"}\NormalTok{), }\KeywordTok{plot}\NormalTok{(Year, Day))}
\end{Highlighting}
\end{Shaded}

Hard to see a trends! \ec

\end{frame}

\begin{frame}[fragile]{With linear model \textbar{} С трендом}
\protect\hypertarget{with-linear-model--}{}

\footnotesize

\begin{Shaded}
\begin{Highlighting}[]
\KeywordTok{require}\NormalTok{(ggplot2)}
\KeywordTok{ggplot}\NormalTok{(}\KeywordTok{subset}\NormalTok{(birches, Site }\OperatorTok{==}\StringTok{ "Centralno-Chernozemnyj"}\NormalTok{), }\KeywordTok{aes}\NormalTok{(Year, Day)) }\OperatorTok{+}\StringTok{ }
\StringTok{    }\KeywordTok{geom_point}\NormalTok{() }\OperatorTok{+}\StringTok{ }\KeywordTok{stat_smooth}\NormalTok{(}\DataTypeTok{method =} \StringTok{"lm"}\NormalTok{) }\OperatorTok{+}\StringTok{ }\KeywordTok{ggtitle}\NormalTok{(}\StringTok{"Centralno-Chernozemnyj"}\NormalTok{)}
\end{Highlighting}
\end{Shaded}

\end{frame}

\begin{frame}[fragile]{Everywhere (and all birches) \textbar{} Всюду (и
все березы - \emph{betula spp.})}
\protect\hypertarget{everywhere-and-all-birches------}{}

\footnotesize

\begin{Shaded}
\begin{Highlighting}[]
\NormalTok{allbirches <-}\StringTok{ }\KeywordTok{subset}\NormalTok{(phenodata, }\KeywordTok{grepl}\NormalTok{(}\StringTok{"Betula"}\NormalTok{, Species) }\OperatorTok{&}\StringTok{ }\NormalTok{Event }\OperatorTok{==}\StringTok{ "onset of blooming"}\NormalTok{)}
\KeywordTok{ggplot}\NormalTok{(allbirches, }\KeywordTok{aes}\NormalTok{(Year, Day, }\DataTypeTok{col =}\NormalTok{ Site)) }\OperatorTok{+}\StringTok{ }\KeywordTok{geom_point}\NormalTok{() }\OperatorTok{+}\StringTok{ }\KeywordTok{stat_smooth}\NormalTok{(}\DataTypeTok{method =} \StringTok{"lm"}\NormalTok{) }\OperatorTok{+}\StringTok{ }\KeywordTok{theme_few}\NormalTok{()}
\end{Highlighting}
\end{Shaded}

\normalsize Negetive trend \textbar{} Отрицательная тенденция?

\end{frame}

\begin{frame}[fragile]{Statistics \textbar{} Статистика}
\protect\hypertarget{statistics-}{}

\bcols\col{.5}

Simple linear model \textbar{} Простая линейная модель

\footnotesize

\begin{Shaded}
\begin{Highlighting}[]
\KeywordTok{lm}\NormalTok{(Day }\OperatorTok{~}\StringTok{ }\NormalTok{Year, }\DataTypeTok{data =}\NormalTok{ allbirches)}
\end{Highlighting}
\end{Shaded}

\footnotesize

NEGATIVE and SIGNIFICANT slope: -0.1 days / year in all study domain.

\pause

\col{.5}

Mixed effects linear model \textbar{} Смешаная линейная модель (без
подробностей - более правильная модель)

\footnotesize

\begin{Shaded}
\begin{Highlighting}[]
\KeywordTok{require}\NormalTok{(lme4)}
\KeywordTok{lmer}\NormalTok{(Day }\OperatorTok{~}\StringTok{ }\NormalTok{Year }\OperatorTok{+}\StringTok{ }\NormalTok{(}\DecValTok{1} \OperatorTok{|}\StringTok{ }\NormalTok{Site), }\DataTypeTok{data =}\NormalTok{ allbirches) }
\end{Highlighting}
\end{Shaded}

\footnotesize

even MORE NEGATIVE and MORE SIGNIFICANT slope: -0.14 days / year.

\ecols

\end{frame}

\begin{frame}[fragile]{What about geographic factors (North / East)?}
\protect\hypertarget{what-about-geographic-factors-north-east}{}

Combining data:

\small

\begin{Shaded}
\begin{Highlighting}[]
\NormalTok{allbirches <-}\StringTok{ }\KeywordTok{merge}\NormalTok{(allbirches, sites, }\DataTypeTok{by=}\StringTok{"SiteID"}\NormalTok{, }\DataTypeTok{all =} \OtherTok{TRUE}\NormalTok{)}
\KeywordTok{ggplot}\NormalTok{(allbirches, }\KeywordTok{aes}\NormalTok{(Year, Day, }\DataTypeTok{col =}\NormalTok{ North, }\DataTypeTok{size =}\NormalTok{ East)) }\OperatorTok{+}\StringTok{ }\KeywordTok{geom_point}\NormalTok{() }\OperatorTok{+}\StringTok{ }\KeywordTok{theme_few}\NormalTok{()}
\end{Highlighting}
\end{Shaded}

\bcols
\col{.6}

\col{.4}
\normalsize
\bi

\I More eastern birches flower later \ldots{} North, less clear.
\I Более восточные березы позже распускают цветы \ldots{} Северные, не
так ясно (). \ei \ecols

\end{frame}

\begin{frame}[fragile]{Statistics \textbar{} Статистика}
\protect\hypertarget{statistics--1}{}

\bc

\small

\begin{Shaded}
\begin{Highlighting}[]
\KeywordTok{lmer}\NormalTok{(Day }\OperatorTok{~}\StringTok{ }\NormalTok{Year }\OperatorTok{+}\StringTok{ }\NormalTok{North }\OperatorTok{+}\StringTok{ }\NormalTok{East }\OperatorTok{+}\StringTok{ }\NormalTok{(Year }\OperatorTok{|}\StringTok{ }\NormalTok{Site.x), }\DataTypeTok{data =}\NormalTok{ allbirches) }
\end{Highlighting}
\end{Shaded}

\small
\ec

\normalsize
\bcols
\col{.5}

All effects (YEAR - NORTH - EAST) significant. About 1.5 days later per
degree north, 1 day later for 4 degrees east. \col{.5} Все переменные
(ГОД - СЕВЕР - ВОСТОК) значимые. Примерно 1.5 дней позже на г.N и один
день на 4 г.E. \ecols

\end{frame}

\begin{frame}{(Preliminary) Conclusions \textbar{} (Предворительные)
Заключения}
\protect\hypertarget{preliminary-conclusions--}{}

\bcols
\col{.3}

Spring - as measured by birch flowering - is very variable, but moving
earlier and earlier in the year (about 2 weeks in 100 years), even
accounting for geographical variation.

\nd\nd\nd

(But need to look at interactions, model comparison, non-linear effects,
etc.)

\col{.3}

Весна - по крайней мере индекс расцветания березы - не смотря на широкий
разброс - крадется все раньше и раньше (пр. 2 недели на 100 лет), даже
если включать географические еффекты.

\nd\nd\nd

(Но надо по-подробнее моделировать).

\col{.4}
\PlotW{1}{birchcatkins.png}

\ecols

\end{frame}

\begin{frame}{Question 2 \textbar{} Вопрос 2}
\protect\hypertarget{question-2--2}{}

\bcols
\col{.3}

\bi

\I How long is fall? \I Сколько осень продолжается? \ei

\col{.6}
\PlotW{1}{autumn.jpg}

\ecols

\end{frame}

\begin{frame}[fragile]{Birches again \textbar{} Опять березы}
\protect\hypertarget{birches-again--}{}

\bcols{}
\col{.6}

Beginning and end of leaf fall:

\scriptsize

\begin{Shaded}
\begin{Highlighting}[]
\NormalTok{birch.start <-}\StringTok{ }\KeywordTok{subset}\NormalTok{(phenodata, }
\NormalTok{                      Event }\OperatorTok{==}\StringTok{ "onset of leaf fall"} \OperatorTok{&}\StringTok{ }\KeywordTok{grepl}\NormalTok{(}\StringTok{"Betula"}\NormalTok{, Species))}
\NormalTok{birch.end <-}\StringTok{ }\KeywordTok{subset}\NormalTok{(phenodata, }
\NormalTok{                    Event }\OperatorTok{==}\StringTok{ "leaf fall end"} \OperatorTok{&}\StringTok{ }\KeywordTok{grepl}\NormalTok{(}\StringTok{"Betula"}\NormalTok{, Species))}
\end{Highlighting}
\end{Shaded}

\normalsize

Merging the two by \texttt{Site}:

\scriptsize

\begin{Shaded}
\begin{Highlighting}[]
\NormalTok{birch.fall <-}\StringTok{ }\KeywordTok{merge}\NormalTok{(birch.start, birch.end, }\DataTypeTok{by =} \KeywordTok{c}\NormalTok{(}\StringTok{'Site'}\NormalTok{,}\StringTok{'Year'}\NormalTok{))}
\end{Highlighting}
\end{Shaded}

\col{.4}

No obvious patterns:

\scriptsize

\begin{Shaded}
\begin{Highlighting}[]
\KeywordTok{with}\NormalTok{(birch.fall, }\KeywordTok{plot}\NormalTok{(Day.x, Day.y, }\DataTypeTok{xlab =} \StringTok{"fall start"}\NormalTok{, }\DataTypeTok{ylab =} \StringTok{"fall end"}\NormalTok{, }\DataTypeTok{col =} \KeywordTok{alpha}\NormalTok{(}\DecValTok{1}\NormalTok{, }\FloatTok{.1}\NormalTok{), }\DataTypeTok{pch =} \DecValTok{19}\NormalTok{))}
\KeywordTok{abline}\NormalTok{(}\DecValTok{0}\NormalTok{,}\DecValTok{1}\NormalTok{, }\DataTypeTok{col=}\DecValTok{2}\NormalTok{, }\DataTypeTok{lwd=}\DecValTok{2}\NormalTok{)}
\end{Highlighting}
\end{Shaded}

\normalsize Note: a few places, fall ended before it began!

\ecols

\end{frame}

\begin{frame}[fragile]{Fall duration against fall start}
\protect\hypertarget{fall-duration-against-fall-start}{}

\scriptsize

\begin{Shaded}
\begin{Highlighting}[]
\NormalTok{birch.fall <-}\StringTok{ }\KeywordTok{mutate}\NormalTok{(birch.fall, }\DataTypeTok{fall.start =}\NormalTok{ Day.x, }\DataTypeTok{fall.end =}\NormalTok{ Day.y, }\DataTypeTok{fall.duration =}\NormalTok{ fall.end }\OperatorTok{-}\StringTok{ }\NormalTok{fall.start)}
\NormalTok{birch.fall <-}\StringTok{ }\KeywordTok{merge}\NormalTok{(birch.fall, sites, }\DataTypeTok{by =} \StringTok{"Site"}\NormalTok{)}
\end{Highlighting}
\end{Shaded}

\bcols
\col{.5}
\small

Against start of fall

\scriptsize

\begin{Shaded}
\begin{Highlighting}[]
\KeywordTok{ggplot}\NormalTok{(birch.fall, }\KeywordTok{aes}\NormalTok{(fall.start, fall.duration, }
                       \DataTypeTok{size=}\NormalTok{East, }\DataTypeTok{col=}\NormalTok{North)) }\OperatorTok{+}\StringTok{ }
\StringTok{    }\KeywordTok{geom_point}\NormalTok{(}\DataTypeTok{alpha =} \FloatTok{0.3}\NormalTok{) }\OperatorTok{+}\StringTok{ }\KeywordTok{theme_few}\NormalTok{() }
\end{Highlighting}
\end{Shaded}

\small

The later the fall starts, the shorter it is. The more East, the more
North, the shorter the fall.

\col{.5}
\small

Against Year:

\scriptsize

\begin{Shaded}
\begin{Highlighting}[]
\KeywordTok{ggplot}\NormalTok{(birch.fall, }\KeywordTok{aes}\NormalTok{(Year, fall.duration, }
                       \DataTypeTok{size=}\NormalTok{East, }\DataTypeTok{col=}\NormalTok{Site)) }\OperatorTok{+}\StringTok{ }
\StringTok{    }\KeywordTok{geom_point}\NormalTok{(}\DataTypeTok{alpha =} \FloatTok{0.3}\NormalTok{) }\OperatorTok{+}\StringTok{ }\KeywordTok{theme_few}\NormalTok{() }
\end{Highlighting}
\end{Shaded}

\small

No obvious trends in time \ecols

\end{frame}

\begin{frame}[fragile]{Statistics}
\protect\hypertarget{statistics}{}

\scriptsize

\begin{Shaded}
\begin{Highlighting}[]
\KeywordTok{lmer}\NormalTok{(fall.duration }\OperatorTok{~}\StringTok{ }\NormalTok{fall.start }\OperatorTok{+}\StringTok{ }\NormalTok{North }\OperatorTok{+}\StringTok{ }\NormalTok{East }\OperatorTok{+}\StringTok{ }\NormalTok{Year }\OperatorTok{+}\StringTok{ }\NormalTok{(}\DecValTok{1}\OperatorTok{|}\NormalTok{Site), }\DataTypeTok{data =}\NormalTok{ birch.fall)}
\end{Highlighting}
\end{Shaded}

\scriptsize

\normalsize 
\bi

\I Lots of significant effects! It seems like fall is getting a bit
LONGER each year, at about the same rate as spring is moving forward.

\I Много значимих еффектов, включая удлиненния осени со временем -
примерно таким же темпом как и весна наступает раньше.\\
\ei

\end{frame}

\begin{frame}[fragile]{Actual conclusions \textbar{} Реальные выводы}
\protect\hypertarget{actual-conclusions--}{}

\bcols
\col{.5}

\bi

\I The package is available to any collaborator (contact me). \I It
should make it relatively EASY to explore and analyze this insanely rich
dataset. \I It will be grow, with more phenological data, meteorological
data, and functions with time. We will try to keep you up to date on its
progress (and availability). \I USE IT! \ei 

\col{.5}
\bi

\I Пакет \texttt{ebfb} доступен любим сотрудником проекта. \I Он создан
для того что-бы упростить аналитическую работу с этим массивном сбором
объедененых данных. \I Пакет будет расти - добавлять данные и
фенологические и метеорологическии, и с новыми функциами. Но мы
постараемся держать коллектив в курсе. \I ИСПОЛЬЗУЕТЕ ЕГО! \ei \ecols

\end{frame}

\begin{frame}{Thanks \textbar{} Спасибо}
\protect\hypertarget{thanks-}{}

\bc
\PlotW{1}{altai.jpg}

(на Алтае) \ec

\end{frame}

\end{document}
